% =============================================================================
\section{Algebraic Analysis}
\label{sec:analysis}
% =============================================================================

\begin{figure}[!htp]
  \centerline{\begin{tikzpicture}[]
    \draw[-] (-1,0) -- (7,0) node[right] {$x$};
    \draw (0,0) node[below left] {$O$};
    \draw[-] (0,-1) -- (0,6) node[above] {$y$};
    \foreach \x in {-1, 1, 2, 3, 4, 5, 6, 7}
      \draw (\x cm,1pt) -- (\x cm,-1pt) node[anchor=north] {$\x$};
      \foreach \y in {-1, 1, 2, 3, 4, 5, 6}
      \draw (1pt,\y cm) -- (-1pt,\y cm) node[anchor=east] {$\y$};
    %
    \coordinate (p21) at (-1,2);
    \coordinate (p22) at (7,0);
    \draw (p21)--(p22) node[pos=0.9,anchor=south,outer sep=3pt]{$L_2$};
    \coordinate (q21) at (5,0.5);
    \coordinate (q22) at (2,1.25);
    \fill[pattern=north east lines,pattern color=red]
      (1,0)--(q22)--(q21)--(5.25,0)--cycle;
    \node[point] (a) at (q21) {};
    \node[point] (b) at (q22) {};
    \draw[thick] (a)--(b);
    %
    \coordinate (p11) at (-1,1);
    \coordinate (p12) at (4,6);
    \draw (p11)--(p12) node[pos=0.9,anchor=south east,outer sep=0pt]{$L_1$};
    \coordinate (q11) at (3,5);
    \coordinate (q12) at (1,3);
    \fill[pattern=north west lines,pattern color=red]
      (1.5,5.5)--(q11)--(q12)--(0,3)--cycle;
    \node[point] (a) at (q11) {};
    \node[point] (b) at (q12) {};
    \draw[thick] (a)--(b);
    %
    \node[point=blue,label={[label distance=-1pt]0:\rotatebox{0}{$p$}}] (p)
      at (6,3) {};
    \node[point=blue,label={[label distance=1pt]90:\rotatebox{0}{$p_2$}}] (p2)
        at (3,1) {};
    \draw[->,>=Stealth] (p)--(p2);
    \draw [decorate,decoration={calligraphic brace,raise=5pt,amplitude=5pt}]
      (p)--(p2) node[midway,anchor=north west,outer sep=3pt]{$d_2$};
    \node[point=blue,label={[label distance=1pt]270:\rotatebox{0}{$p_1$}}] (p1)
      at (1.5,3.5) {};
    \draw[->,>=Stealth] (p)--(p1);
    \draw [decorate,decoration={calligraphic brace,raise=5pt,amplitude=5pt}]
      (p1)--(p) node[midway,anchor=south,outer sep=10pt]{$d_1$};
    \pic [draw,->,"$\alpha$",angle eccentricity=1.5] {angle=p1--p--p2};
    \node [point=green] at (-0.1875,1.8125) {};
  \end{tikzpicture}}
  \caption{Local view of the problem.}
  \label{fig:analysis}
\end{figure}

We start with an algebraic analysis of the problem. Here, we concentrate at a local view of the problem, where we only consider two walls (edges) of obstacles and ignore everything else; see Figure~\ref{fig:analysis}. Let $L_1: a_1 x + b_1 y + c_1 = 0$ and $L_2: a_2 x + b_2 y + c_2 = 0$ denote the two underlying lines of the two edges of obstacles hit by the two measuring rays, respectively. Let $p = (x,y)$ denote a point in the workspace our sensor could be located at. Let $p_1 = (x_1,y_1)$ denote the point on $L_1$ hit by the first measuring ray, and similarly, let $p_2 = (x_2,y_2)$ denote the point on $L_2$ hit by the second measuring ray. Employing the law of cosine (Equation~\ref{eq:loc}), the following equations must be satisfied:
\begin{gather}
  a_1\cdot x_1 + b_1\cdot y_1 + c1 = 0\\
  a_2\cdot x_2 + b_2\cdot y_2 + c_2 = 0\\
  |p - p_1| = d_1\\
  |p - p_2| = d_2\\
  d_1^2 + d_2^2 + 2\cdot \cos(\alpha)\cdot d_1\cdot d_2 = |p_1 - p_2|^2\label{eq:loc}
\end{gather}
In the degenerate case, where $L_1$ and $L_2$ are parallel (or $L_1 == L_2$) the loci of all points that satisfy the above equations form a line. This line must be trimmed to a segment according to the actual edge lengths and interiors of the obstacles. In all other cases the loci of points form two ellipses that correspond to clockwise and counter clockwise rotations. The ellipse that corresponds to the clockwise location is discarded and the other must be trimmed to obtain that actual possible locations.

Solving the above non-linear equation system is messy. We can add some constraints to the above system without compromising. First, we translate the scene, by setting $c_1 = 0$, $c_2 = 0$, coercing the intersection point of $L_1$ and $L_2$ to be at the origin. Then, we rotate the scene, setting $a_1 = 0$, coercing $L_1$ to lie on the $x$-axis. Once we find the desired elliptic arc in our transformed space, we can apply an inverse rotation followed by an inverse translation to obtain the elliptic arc in the original space.

First, we handle the simple case, where $b_2 = 0$. Recall that $L_1$ lies on the $x$-axis. This additional constraint implies that $L_2$ lies on the $y$-axis. ($L_1$ and $L_2$ are orthogonal.) Manipulating the system equation above and employing Matlab to simplify the equations above, we obtain the single equation:
\begin{equation}
  d_1^4\cdot x^4 + (4\cdot k^2 - 2\cdot d_1^2\cdot d_2^2)\cdot x^2\cdot y^2 - 2\cdot d_1^2\cdot k^2\cdot x^2 + d_2^4\cdot y^4 + k^4 - 2\cdot d_2^2\cdot k^2\cdot y^2 = 0\label{eq:polynomial-orthogonal-x-aligned},
\end{equation}
where $k = \cos(\alpha)\cdot d_1\cdot d_2$. Let $P_1$ denote the bivariate polynomial on the left hand side of Equation~\ref{eq:polynomial-orthogonal-x-aligned}. The zero set of $P_1$ represents the two desired ellipses. Employing Matlab yet again to factorize the polynomial $P_1$, we get: $P_1: (A_1\cdot x^2 + B_1\cdot y^2 + C_1\cdot x\cdot y + D_1)\cdot (A_2\cdot x^2 + B_2\cdot y^2 + C_2\cdot x\cdot y +D_2)$, where
\begin{minipage}[c]{0.5\linewidth}
\begin{align*}
  A_1 &= d_1^2\\
  B_1 &= d_2^2\\
  C_1 &= 2*(d_1^2\cdot d_2^2 - k^2)(1/2)\\
  D_1 &= -k^2
\end{align*}
\end{minipage}
\begin{minipage}[c]{0.5\linewidth}
\begin{align*}
  A_2 &= d_1^2\\
  B_2 &= d_2^2\\
  C_2 &= -2*(d_1^2\cdot d_2^2 - k^2)(1/2)\\
  D2 &= -k^2
\end{align*}
\end{minipage}
You can visualize the ellipses and how they dynamically change as a
consequence of changing the parameters $d_1$, $\alpha$, $d_2$ using
the GeoGebra\footnote{\url{https://www.geogebra.org}} online tool;
download the GeoGebra data
file~\url{https://www.geogebra.org/calculator/rvqwxqcd} and upload in
GeoGebra. Second, we denote $m_2 = \frac{a_2}{b_2}$, assuming the
lines are not parallel. The obtained polynomial, $P_2$, in this case
is much more complex; see~\ref{ssec:derivation-p2}. Use the GeoGebra data file
\url{https://www.geogebra.org/calculator/ferztmgh} to
visualize the ellipses in this case.
