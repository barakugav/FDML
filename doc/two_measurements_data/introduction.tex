% =============================================================================
\section{Introduction}
\label{sec:intro}
% =============================================================================
In this variant a query consists of three real numbers $d_1$, $\alpha$, $d_2$ describing the following sequence of events: The sensor at its original state obtained the distance reading $d_1$, then the sensor was rotated (without translating) by $\alpha$ radians counterclockwise, and then it obtained a second distance reading $d_2$. A more complicated problem allows for a translation of the robot before the second measurement is taken. In this variant a query consists of four real numbers $d_1$, $\alpha$, $t$, $d_2$, where $t$ denotes a translation vector in the plane. If $t \neq 0$, the two measurements can be taken simultaneously. This is possible in practice, if two distinct sensors are at our disposal. We made experiments with a real robot equipped with two sensors.
