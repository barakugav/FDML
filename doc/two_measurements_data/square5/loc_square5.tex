\documentclass{standalone}
\usepackage{xparse}
\usepackage{tikz}
\tikzset{%
  point/.style={circle,inner sep=0.5pt,minimum size=0.5pt,draw,fill=#1},
  point/.default=red
}
\definecolor{c0}{rgb}{0.2,0.4,0.67}
\definecolor{c1}{rgb}{0.67,0.4,0.12}
\definecolor{c2}{rgb}{0.53,0.6,0.13}
\definecolor{c3}{rgb}{0.53,0.53,0.4}
\NewDocumentCommand\witness{O{a}O{b}O{c}O{c2}}{%
  \node[point=#4] (#1) at (0,1) {};
  \node[point=#4] (#2) at (1,0) {};
  \node[point=#4] (#3) at (1,1) {};
  \draw[#4] (#1)--(#2)--(#3);
}
\NewDocumentCommand\crvb{}{%
  \def\a{-0.618034}
  \def\b{-1.61803}
  \def\c{.52573}
  \def\s{.85065}
  \draw[c1,parametric,variable=\t,smooth,
    domain=1.66:1.89,
    samples=60]
  plot
  ({\a*cos(\t r)*\c-\b*sin(\t r)*\s},{\a*cos(\t r)*\s+\b*sin(\t r)*\c});
}
\NewDocumentCommand\obs{}{%
  \coordinate (q0) at (0.76,0.1);
  \coordinate (q1) at (3.05,0.1);
  \coordinate (q2) at (3.05,0.2);
  \coordinate (q3) at (0.76,0.2);
  \filldraw[fill=white,draw=c0] (q0) -- (q1) -- (q2) -- (q3) -- cycle;%
  \node[point=c0] at (q0) {};
  \node[point=c0] at (q1) {};
  \node[point=c0] at (q2) {};
  \node[point=c0] at (q3) {};
}

\begin{document}
\begin{tikzpicture}[]
  \coordinate (p0) at (-2,-2);
  \coordinate (p1) at (2,-2);
  \coordinate (p2) at (2,2);
  \coordinate (p3) at (-2,2);
  \filldraw[fill=lightgray,draw=c0] (p0) -- (p1) -- (p2) -- (p3) -- cycle;%
  \node[point=c0] at (p0) {};
  \node[point=c0] at (p1) {};
  \node[point=c0] at (p2) {};
  \node[point=c0] at (p3) {};
  %
  \begin{scope}[shift={(-2,-2)},rotate=0]\obs\end{scope}
  \begin{scope}[shift={(2,-2)},rotate=90]\obs\end{scope}
  \begin{scope}[shift={(2,2)},rotate=180]\obs\end{scope}
  \begin{scope}[shift={(-2,2)},rotate=270]\obs\end{scope}
  %
  \coordinate (r0) at (-0.95,-0.95);
  \coordinate (r1) at (0.95,-0.95);
  \coordinate (r2) at (0.95,0.95);
  \coordinate (r3) at (-0.95,0.95);
  \filldraw[fill=white,draw=c0] (r0) -- (r1) -- (r2) -- (r3) -- cycle;%
  \node[point=c0] at (q0) {};
  \node[point=c0] at (q1) {};
  \node[point=c0] at (q2) {};
  \node[point=c0] at (q3) {};
  %%% Witnesses
  \begin{scope}[shift={(-0.586,-1.293)},rotate=135]\witness\end{scope}
  \begin{scope}[shift={(-0.592,-1.234)},rotate=140]\witness\end{scope}
  \node[point=c1] at (b) {};
  \begin{scope}[rotate=90]
    \begin{scope}[shift={(-0.592,-1.234)},rotate=140]\node[point=c1] at (1,0) {};\end{scope}
  \end{scope}
  \begin{scope}[rotate=180]
    \begin{scope}[shift={(-0.592,-1.234)},rotate=140]\node[point=c1] at (1,0) {};\end{scope}
  \end{scope}
  \begin{scope}[rotate=270]
    \begin{scope}[shift={(-0.592,-1.234)},rotate=140]\node[point=c1] at (1,0) {};\end{scope}
  \end{scope}
  %
  \begin{scope}[shift={(-0.601,-1.399)},rotate=127]\witness\end{scope}
  \node[point=c1] at (b) {};
  \begin{scope}[rotate=90]
    \begin{scope}[shift={(-0.601,-1.399)},rotate=127]\node[point=c1] at (1,0) {};\end{scope}
  \end{scope}
  \begin{scope}[rotate=180]
    \begin{scope}[shift={(-0.601,-1.399)},rotate=127]\node[point=c1] at (1,0) {};\end{scope}
  \end{scope}
  \begin{scope}[rotate=270]
    \begin{scope}[shift={(-0.601,-1.399)},rotate=127]\node[point=c1] at (1,0) {};\end{scope}
  \end{scope}
  %% Solution
  \begin{scope}[shift={(-2,2)},rotate=0]\crvb\end{scope}
  \begin{scope}[shift={(-2,-2)},rotate=90]\crvb\end{scope}
  \begin{scope}[shift={(2,-2)},rotate=180]\crvb\end{scope}
  \begin{scope}[shift={(2,2)},rotate=270]\crvb\end{scope}
  %%
\end{tikzpicture}
\end{document}
