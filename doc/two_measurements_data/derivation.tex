% =============================================================================
\section{Derivation}
\label{sec:derivation}
% =============================================================================
We repeat the system equation in the general case:
\begin{align}
  a_1 x_1 + b_1 y_1 + c_1 &= 0\\
  a_1 x_2 + b_1 y_2 + c_1 &= 0\\
  (x - x_1)^2 + (y - y_1)^2 &= d_1^2\label{eq:1}\\
  (x - x_2)^2 + (y - y_2)^2 &= d_2^2\label{eq:2}\\
  (x_2 - x_1)^2 + (y_2 - y_1)^2 &= d_1^2 + d_2^2 - 2 c d_1 d_2 = d_1^2 + d_2^2 - 2 k\label{eq:3}
\end{align}

% -----------------------------------------------------------------------------
\subsection{Constraining the Slopes of Both Lines}
\label{ssec:derivation-constraining-slopes}
% -----------------------------------------------------------------------------
We set $y_1 = 0$ and $x_2 = 0$, coercing $L_1$ and $L_2$ to lie on the
$x$- and $y$-axes, respectively.
Equations~\ref{eq:1},~\ref{eq:2}, and~\ref{eq:3} reduce to:
\begin{align}
  (x - x_1)^2 + y^2 &= d_1^2\label{eq:d1-oa}\\
  x^2 + (y - y_2)^2 &= d_2^2\label{eq:d2-oa}\\
  x_1^2 + y_2^2 &= d_1^2 + d_2^2 - 2 k\label{eq:d-oa}
\end{align}

\noindent
We substitute $d_1$ and $d_2$ in Equation~\ref{eq:d-oa} and obtain the single equation:
\begin{align}
  x^2 + y^2 - x x_1 - y y_2 - k &= 0\label{eq:s-oa}
\end{align}

\noindent
We introduce the symbols $s_1$ and $s_2$, and get
\begin{align*}
  x_1 &= x \pm \sqrt{s_1}\\
  y_2 &= y \pm \sqrt{s_2},
\end{align*}
where $s_1 = d_1^2 - y^2$ and $s_2 = d_2^2 - x^2$ based on equations~\ref{eq:d1-oa} and~\ref{eq:d2-oa}.

\noindent
We substitute $x_1$ and $x_2$ in Equation~\ref{eq:s-oa} and get:
\begin{align*}
  x^2 + y^2 - x (x \pm \sqrt{s_1}) - y (y \pm \sqrt{s_2}) - k &= 0
\end{align*}

\noindent
We expand and exchange and get:
\begin{align*}
  \pm x \sqrt{s_1} \pm y \sqrt{s_2} &= k
\end{align*}

\noindent
We raise to the power of two each side and get:
\begin{align*}
  x^2 s_1 + y^2 s_2 + 2 x y \sqrt{s_1 s_2} &= k^2
\end{align*}

\noindent
We exchange and get:
\begin{align*}
  2 x y \sqrt{s_1 s_2} &= k^2 - x^2 s_1 - y^2 s_2
\end{align*}

\noindent
We raise again and get:
\begin{align*}
  4 x^2 y^2 s_1 s_2 &= (k^2 - x^2 s_1 - y^2 s_2)^2
\end{align*}

\noindent
We expand and get:
\begin{align*}
  4 x^2 y^2 s_1 s_2 &= k^4 + x^4 s_1^2 + y^4 s_2^2 - 2 k^2 x^2 s_1 - k^2 y^2 s_2 + 2 x^2 y^2 s_1 s_2
\end{align*}

\noindent
We exchange and substitute $s_1$ and $s_2$ and get:
\begin{align*}
  k^4 + x^4 (d_1^2 - y^2)^2 + y^4 (d_2^2 - x^2)^2 - 2 k^2 x^2 (d_1^2 - y^2) - k^2 y^2 (d_2^2 - x^2) - 2 x^2 y^2 (d_1^2 - y^2) (d_2^2 - x^2) &= 0\\
\end{align*}

\noindent
We expand and regroup and get:
\begin{align}
  d_1^4 x^4 + y^2 x^2 (4 k^2 - 2 d_1^2 d_2^2) - 2 d_1^2 k^2 x^2 + d_2^4 y^4 - 2 d_2^2 k^2 y^2 + k^4 &= 0\label{eq:oa}
\end{align}

% -----------------------------------------------------------------------------
\subsection{Constraining the Intersection Point}
\label{ssec:derivation-constraining-origin}
% -----------------------------------------------------------------------------
\setlength{\parskip}{1ex}
%\noindent
We set $c_1 = c_2 = 0$, coercing the intersection point of $L_1$ and $L_2$ to coincide with the origin, and denote $m_1 = \frac{a_1}{b_1}$ and $m_2 = \frac{a_2}{b_2}$; we get $y_1 = m_1 x_1$ and $y_2 = m_2 x_2$.

\noindent
We substitute $y_1$ and $y_2$ in equations~\ref{eq:1}, \ref{eq:2}, and \ref{eq:3} and get:
\begin{align}
  (x - x_1)^2 + (y - m_1 x_1)^2 & = d_1^2\\
  (x - x_2)^2 + (y - m_2 x_2)^2 & = d_2^2\\
  (x_2 - x_1)^2 + (m_2 x_2 - m_1 x_1)^2 & = d_1^2 + d_2^2 - 2 k
\end{align}

\noindent
We expand and regroup and get:
\begin{align}
  %x^2 + x_1^2 - 2 x x_1 + y^2 + m_1^2x_1^2 - 2 m_1 y x_1 &= d_1^2\\
  (1 + m_1^2) x_1^2 - 2 (x + m_1 y) x_1  + x^2 + y^2  &= d_1^2\label{eq:d1}\\
  (1 + m_2^2) x_2^2 - 2 (x + m_2 y) x_2  + x^2 + y^2  &= d_2^2\label{eq:d2}\\
  % x_2^2 + x_1^2 - 2 x_1 x_2 + m_1^2 x_1^2 + m_2^2 x_2^2 - 2 m_1 m_2 x_1 x_2 &= d_1^2 + d_2^2 - 2 k\\
  (1 + m_1^2) x_1^2 + (1 + m_2^2) x_2^2 - 2 (1 + m_1 m_2) x_1 x_2 &= d_1^2 + d_2^2 - 2 k\label{eq:d}
\end{align}

\noindent
We substitute $d_1$ and $d_2$ in Equation~\ref{eq:d} and obtain the single equation:
\begin{align}
  % 2 (1 + m_1 m_2) x_1 x_2 - 2 (x + m_1 y) x_1 - 2 (x + m_2 y) x_2 + 2 x^2 + 2 y^2 &= 2 k\\
  (1 + m_1 m_2) x_1 x_2 - (x + m_1 y) x_1 - (x + m_2 y) x_2 + x^2 + y^2 - k &= 0\label{eqs}
\end{align}

\noindent
We introduce the symbols $r_1$, $s_1$, $r_2$, and $s_2$, and get
\begin{align*}
  x_1 &= r_1 \pm \sqrt{s_1}\\
  x_2 &= r_2 \pm \sqrt{s_2},
\end{align*}
where
\begin{align*}
  r_1 &= ((x + m_1 y) / (1 + m_1^2))\\
  s_1 &= (((x + m_1 y)^2 - (1 + m_1^2) (x^2 + y^2 - d_1^2)) / (1 + m_1^2)^2)\\
  r_2 &= ((x + m_2 y) / (1 + m_2^2))\\
  s_2 &= (((x + m_2 y)^2 - (1 + m_2^2) (x^2 + y^2 - d_2^2)) / (1 + m_2^2)^2)
\end{align*}

\noindent
We substitute $x_1$ and $x_2$ in Equation~\ref{eqs} and get:
\begin{align*}
(1 + m_1 m_2) (r_1 + \sqrt{s_1}) (r_2 + \sqrt{s_2}) - (x + m_1 y) (r_1 + \sqrt{s_1}) - (x + m_2 y) (r_2 + \sqrt{s_2}) + x^2 + y^2 - k &= 0
\end{align*}

\noindent
We expand and regroup and get:
\begin{multline*}
  (1 + m_1 m_2) r_1 r_2
  + (1 + m_1 m_2) r_1 \sqrt{s_2}
  + (1 + m_1 m_2) r_2 \sqrt{s_1}
  + (1 + m_1 m_2) \sqrt{s_1 s_2})\\
  - \sqrt{s_1} (x + m_1 y)
  - \sqrt{s2} (x + m_2 y)
  - r_1 (x + m_1 y) - r_2 (x + m_2 y) + x^2 + y^2 - k = 0
\end{multline*}

\noindent
We exchange and get:
\begin{align*}
  \sqrt{s_1} (r_2 \ell - x - m_1 y) + \sqrt{s_2} (r_1 \ell - x - m_2 y) &=
  r1 (x + m_1 y) + r_2 (x + m_2 y) - x^2 - y^2 + k - \ell r_1 r_2 - \sqrt{s_1 s_2} \ell,
\end{align*}
where $\ell = 1 + m_1 m_2$.

\noindent
We raise to the power of two each side and get:
\begin{multline*}
  s_1 (r_2 \ell - x - m_1 y)^2 + s_2 (r_1 \ell - x - m_2 y)^2 +
      2 \sqrt{s_1 s_2} (r_2 \ell - x - m_1 y) (r_1 \ell - x - m_2 y) = \\
  (r_1 (x + m_1 y) + r_2 (x + m_2 y) - x^2 - y^2 + k - \ell (r_1 r_2))^2 + \ell^2 s_1 s_2 -\\
      2 \sqrt{s_1 s_2} \ell (r_1 (x + m_1 y) + r_2 (x + m_2 y) - x^2 - y^2 + k - \ell (r_1 r_2))
\end{multline*}

\noindent
We exchange and get:
\begin{multline*}
  2 \sqrt{s_1 s_2} ((r_2 \ell - x - m_1 y) (r_1 \ell - x - m_2 y) +
    \ell (r_1 (x + m_1 y) + r_2 (x + m_2 y) - x^2 - y^2 + k - \ell r_1 r_2)) =\\
  (r_1 (x + m_1 y) + r_2 (x + m_2 y) - x^2 - y^2 + k - \ell r_1 r_2)^2 + \ell^2 s_1 s_2 -
      s_1 (r_2 \ell - x - m_1 y)^2 - s_2 (r_1 \ell - x - m_2 y)^2
\end{multline*}

\noindent
We raise again and get:
\begin{multline}
  4 s_1 s_2 ((r_2 \ell - x - m_1 y) (r_1 \ell - x - m_2 y) + \ell (r_1 (x + m_1 y) + r_2 (x + m_2 y) - x^2 - y^2 + k - \ell r_1 r_2))^2 =\\
  ((r_1 (x + m_1 y) + r_2 (x + m_2 y) - x^2 - y^2 + k - \ell r_1 r_2)^2 + \ell^2 s_1 s_2 -
    s_1 (r_2 \ell - x - m_1 y)^2 - s_2 (r_1 \ell - x - m_2 y)^2)^2\label{eq:g}
\end{multline}

% -----------------------------------------------------------------------------
\subsection{Constraining the Slope of One Line}
\label{ssec:derivation-constraining-slope}
% -----------------------------------------------------------------------------
We set $m_1 = 0$, which implies that $\ell = 1$, in
Equation~\ref{eq:g}, coercing $L_1$ to lie on the $x$-axis, and get:
\begin{multline*}
  4 s_1 s_2 ((r_2 - x - m_1 y) (r_1 - x - m_2 y) + (r_1 (x + m_1 y) + r_2 (x + m_2 y) - x^2 - y^2 + k - r_1 r_2))^2 =\\
  ((r_1 (x + m_1 y) + r_2 (x + m_2 y) - x^2 - y^2 + k - r_1 r_2)^2 + s_1 s_2 -
    s_1 (r_2 - x - m_1 y)^2 - s_2 (r_1 - x - m_2 y)^2)^2
\end{multline*}
We simplify the resutimg equation using Matlab. We obtain the
bivariate polynomial $P_2$ (see below), the zero set of which
represents the points satisfying the equation. We employ Matlab yet
again to factorize the polynomial $P_2$. The Matlab script is
available at
\url{http://acg.cs.tau.ac.il/projects/in-house-projects/localization-with-few-distance-measurements/solution.m}.

The bivariate polynomial obtained by simplifying the system equation
in the case where $L_1$ lies on the $x$-axis and and the intersection
point of $L_1$ and $L_2$ coincides with the origin follows.

\begin{equation}\label{eq:a}
  \begin{split}
  & P_2 && : && 4 \cdot k^2 \cdot y^4
    - 4 \cdot k^3 \cdot y^2
    + k^4 + d_1^4 \cdot d_2^4
    + 2 \cdot k^4 \cdot m_2^2
    + k^4 \cdot m_2^4
    + d_1^4 \cdot y^4
    + d_2^4 \cdot y^4 -\\
  & && && 4 \cdot d_1^2 \cdot k \cdot y^4
    - 4 \cdot d_2^2 \cdot k \cdot y^4
    - 2 \cdot d_1^2 \cdot d_2^2 \cdot k^2
    + 2 \cdot d_1^2 \cdot d_2^2 \cdot y^4
    - 2 \cdot d_1^2 \cdot d_2^4 \cdot y^2 -\\
  & && && 2 \cdot d_1^4 \cdot d_2^2 \cdot y^2
    + 2 \cdot d_1^2 \cdot k^2 \cdot y^2
    + 2 \cdot d_2^2 \cdot k^2 \cdot y^2
    + d_1^4 \cdot m_2^4 \cdot x^4
    + 2 \cdot d_2^4 \cdot m_2^2 \cdot y^4 +\\
  & && && d_2^4 \cdot m_2^4 \cdot y^4
    - 4 \cdot k^3 \cdot m_2^2 \cdot y^2
    + 4 \cdot k^2 \cdot m_2^2 \cdot y^4
    + 4 \cdot d_1^2 \cdot d_2^2 \cdot k \cdot y^2 - \\
  & && && 4 \cdot d_2^2 \cdot k \cdot m_2^2 \cdot y^4
    - 4 \cdot d_1^4 \cdot m_2^3 \cdot x^3 \cdot y
    - 8 \cdot k^2 \cdot m_2^3 \cdot x \cdot y^3
    - 2 \cdot d_1^2 \cdot d_2^2 \cdot k^2 \cdot m_2^2 +\\
  & && && 4 \cdot k^3 \cdot m_2 \cdot x \cdot y
    - 2 \cdot d_1^4 \cdot d_2^2 \cdot m_2^2 \cdot x^2
    - 2 \cdot d_1^2 \cdot d_2^2 \cdot m_2^2 \cdot y^4
    - 2 \cdot d_1^2 \cdot d_2^4 \cdot m_2^2 \cdot y^2+\\
  & && && 2 \cdot d_1^2 \cdot k^2 \cdot m_2^2 \cdot x^2
    - 2 \cdot d_1^2 \cdot k^2 \cdot m_2^4 \cdot x^2
    - 2 \cdot d_1^2 \cdot k^2 \cdot m_2^2 \cdot y^2
    - 2 \cdot d_2^2 \cdot k^2 \cdot m_2^4 \cdot y^2 +\\
  & && && 6 \cdot d_1^4 \cdot m_2^2 \cdot x^2 \cdot y^2
    + 4 \cdot k^2 \cdot m_2^2 \cdot x^2 \cdot y^2
    + 4 \cdot k^2 \cdot m_2^4 \cdot x^2 \cdot y^2
    - 4 \cdot d_1^4 \cdot m_2 \cdot x \cdot y^3 -\\
  & && && 8 \cdot k^2 \cdot m_2 \cdot x \cdot y^3
    + 4 \cdot k^3 \cdot m_2^3 \cdot x \cdot y
    + 4 \cdot d_1^2 \cdot d_2^2 \cdot m_2^3 \cdot x \cdot y^3
    - 12 \cdot d_1^2 \cdot k \cdot m_2^2 \cdot x^2 \cdot y^2 +\\
  & && && 4 \cdot d_1^4 \cdot d_2^2 \cdot m_2 \cdot x \cdot y
    - 4 \cdot d_1^2 \cdot k^2 \cdot m_2 \cdot x \cdot y
    + 12 \cdot d_1^2 \cdot k \cdot m_2 \cdot x \cdot y^3 +\\
  & && && 4 \cdot d_2^2 \cdot k \cdot m_2 \cdot x \cdot y^3
    + 2 \cdot d_1^2 \cdot d_2^2 \cdot m_2^2 \cdot x^2 \cdot y^2
    - 2 \cdot d_1^2 \cdot d_2^2 \cdot m_2^4 \cdot x^2 \cdot y^2 -\\
  & && && 4 \cdot d_1^2 \cdot d_2^2 \cdot m_2 \cdot x \cdot y^3
    + 4 \cdot d_1^2 \cdot k^2 \cdot m_2^3 \cdot x \cdot y
    + 4 \cdot d_1^2 \cdot k \cdot m_2^3 \cdot x^3 \cdot y +\\
  & && && 4 \cdot d_2^2 \cdot k \cdot m_2^3 \cdot x \cdot y^3
    + 8 \cdot d_1^2 \cdot d_2^2 \cdot k \cdot m_2^2 \cdot y^2
    - 8 \cdot d_1^2 \cdot d_2^2 \cdot k \cdot m_2^3 \cdot x \cdot y -\\
  & && && 4 \cdot d_1^2 \cdot d_2^2 \cdot k \cdot m_2 \cdot x \cdot y\\
  & && =\quad && (A_1\cdot x^2 + B_1\cdot y^2 + C_1\cdot x \cdot y + D_1)\cdot
    (A_2\cdot x^2 + B_2\cdot y^2 + C_2\cdot x \cdot y + D_2),\\
  \end{split}
\end{equation}

%
where\\
\begin{minipage}[c]{0.5\textwidth}
\begin{align*}
  A_1 &= d_1^2\cdot m_2^2\\
  B_1 &= d_1^2+d_2^2-2\cdot k+d_2^2\cdot m_2^2+2\cdot m_2\cdot e\\
  C_1 &= 2\cdot m_2\cdot (k - d_1^2 - m_2\cdot e)\\
  D_1 &= k^2 - d_1^2\cdot d_2^2 - k^2\cdot m_2^2 - 2\cdot k\cdot m_2\cdot e\\
\end{align*}
\end{minipage}
%
\begin{minipage}[c]{0.5\textwidth}
\begin{align*}
  A_2 &= d_1^2\cdot m_2^2\\
  B_2 &= d_1^2 - 2\cdot k + d_2^2 + d_2^2\cdot m_2^2 - 2\cdot m_2\cdot  e\\
  C_2 &= 2\cdot m_2\cdot (k - d_1^2 + m_2\cdot  e)\\
  D_2 &= k^2 - d_1^2\cdot d_2^2 - k^2\cdot m_2^2 + 2\cdot k\cdot m_2\cdot e\\
\end{align*}
\end{minipage}
and $e = \sqrt{d_1^2\cdot d_2^2 - k^2}$.

%% % -----------------------------------------------------------------------------
%% \subsection{Constraining the Two Lines to be Orthogonal}
%% \label{ssec:derivation-constraining-orthogonal}
%% % -----------------------------------------------------------------------------
%% If the $L_1$ and $L_2$ are orthogonal $m_1 m_2 = -1$. It implies that $\ell = 0$.
%% We set $\ell = 0$ in Equation~\ref{eq:g}, coercing $L_1$ and $L_2$ to be
%% orthogonal, and get:
%% \begin{multline*}
%%   4 s_1 s_2 ((x + m_1 y) (x + m_2 y))^2 =
%%   ((r_1 (x + m_1 y) + r_2 (x + m_2 y) - x^2 - y^2 + k)^2 - s_1 (x + m_1 y)^2 - s_2 (x + m_2 y)^2)^2
%% \end{multline*}

%% We simplify the resutimg equation using Matlab. We obtain the
%% bivariate polynomial $P$, the zero set of which represents the
%% points satisfying the equation. We employ Matlab yet again to
%% factorize the polynomial $P$; see Section~\ref{ssec:derivation-po}
%% for the final expressions.
