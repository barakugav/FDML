\documentclass[a4paper,12pt]{extarticle}
\usepackage[utf8]{inputenc}
\usepackage[a4paper, margin=1in]{geometry}
\usepackage{amsmath}
\usepackage{graphicx}
\graphicspath{ {./img/} }

\title{Robot Localization with Few Distance Measurements}
\author{}
\date{January 2022}

\begin{document}

\maketitle

\subsubsection*{The Question}

We are given a point robot $P$ which is known to be in the interior
of some polygonal room $Q$ with $n$ vertices. The robot is equipped
with a depth sensor which evaluates the distance to the visible obstacle
in the pointed direction. Describe a data structure that can be constructed
during preprocessing of the room and can be used to efficiently calculate
all the points the robot might think he is after a single depth measurement
from an unknown position and rotation.

\subsubsection*{Motivation}

Such scenario can be applicable in real world cases, for example:
a rover lands on Mars, a map of whose terrain is available to it.
It looks about its position, and then infers its exact position on
the Martian surface. Another application comes from robots that follow
a planned path through a scene: the control systems that guide such
a robot along the planned path gradually accumulate errors due to
mechanical drift. Thus it is desirable to use localization from time
to time to verify the actual position of the robot in the map, and
apply corrections as necessary to return it to the planned path.

\newpage{}

\subsubsection*{Our solution}

Given a measurement $d$, if we assume the robot orientation is exactly in the direction of the positive $y$ axis, we can perform a vertical decomposition of the room, and easily calculate for each cell in the decomposition arrangement if there are any points in the cell that are at distance $d$ from the top cell edge, and union all the results. To support different orientations of the robot, one might perform this method to multiple versions of the scene, each rotated at different angle, and union all the results. In each rotation, different cells will be created during the decomposition. Some of these cells are very similar to each other. We can describe each cell only by its top and bottom edges, and the two vertices that defines the two artificial edges. With this description, there are only $O(n^{2})$ unique cells, and we can calculate each cell's exact boundary as a function of the rotation angle. From now on, we will only rotate the ``vertical'' lines used for the decomposition, not the scene itself.

A vertical decomposition cell $C$ is identified by its top and bottom original edges of the scene, $e^{C}_{t},e^{C}_{b}$, and the left and right vertices, $v^{C}_{l},v^{C}_{r}$, that define the left and right artificial edges (parallel to the rotated $y$-axis). The two limiting vertices, $v^{C}_{l},v^{C}_{r}$, may be endpoints of $e^{C}_{t}$, $e^{C}_{b}$, or some other vertex.

\begin{figure}[h]
    \centering
    \includegraphics[scale=0.31]{img/limiting_vertices.png}
    \caption{limiting left and right vertices}
\end{figure}

We can identify these cells by performing a rotational plane sweep originated at all vertices simultaneously, each vertex will be an origin of an (imaginary) ray, all rays are rotated together. Each ray maintains a balanced binary search tree that contains all the edges that the ray intersect (similar to regular rotational plane sweep). An event occurs when one of the rays hits a vertex, and we calculate all $O(n^{2})$ events in advance and sort them. In addition, each ray stores at each angle the cell from both it's sides, allowing us to keep track of all the existing cells with the current orientation. During handling an event of two colinear vertices, if they are visible to one another, a cell should be created or terminated.

There are 2 types of events:
\begin{itemize}
  \item Two vertices of the same edge: The triangle cell that contained both vertices is "squeezed" and terminated, and a new triangle cell is created. Also the cell that share the cell artificial edge also terminate and a new one is created due to change of one of the limiting vertices. 2 terminations and 2 creations total.
  \item Two vertices of different edge: The middle cell is "squeezed" and terminated, and a new middle cell is created. The cells from both sides of the connecting ray terminate and new ones are created due to change in the limiting vertices. 3 terminations and 3 creations total.
\end{itemize}
In both event types, the top and bottom edges and left and right vertices are known locally by the rays or the terminated cells. During handling of an event we terminate and create constant number of cells, there are $O(n^{2})$ events. To start the process, regular vertices decomposition is performed.
In total, we can identify these $O(n^{2})$ cells and their valid angle intervals in $O(n^{2}\log n)$ time and $O(n^{2})$ space.

\begin{figure}[h]
    \centering
    \includegraphics[scale=0.21]{img/sweep1.png}
    \includegraphics[scale=0.21]{img/sweep2.png}
    \caption{Event of vertices of the same edge example. Rays are rotated counterclockwise, the event vertices are bold. (each column is an example)}
\end{figure}
\begin{figure}[h]
    \centering
    \includegraphics[scale=0.21]{img/sweep3.png}
    \includegraphics[scale=0.21]{img/sweep4.png}
    \includegraphics[scale=0.21]{img/sweep5.png}
    \includegraphics[scale=0.21]{img/sweep6.png}
    \caption{Event of vertices of the different edge example. Rays are rotated counterclockwise, the event vertices are bold. (each column is an example)}
\end{figure}

The top left and top right vertices (not necessary exists in the original scene) of a cell $C$ both lay on $e^{C}_{t}$. We can calculate their $x$ values as a function of $\theta$, denoted $x^{C}_{tl}(\theta),x^{C}_{tr}(\theta)$.

\begin{figure}[h]
    \centering
    \includegraphics[scale=0.35]{img/xtl_xtr.png}
    \caption{$x$ value of top left and top right vertices of a cell}
\end{figure}

Denote the size of the opening at angle $\theta$ at $x$ of the top edge, between the top and bottom edges of a cell by $O^{C}(\theta,x)=\frac{(e^{C}_{t}(x)-e^{C}_{b}(x)) \sin \alpha}{\cos (\alpha - \theta)}$ ($\alpha$ is the angle of the bottom edge, see calculation section).

\begin{figure}[h] \begin{center}
    \includegraphics[scale=0.3]{img/opening.png}
    \caption{the opening size $O$ as a function of $x$ and $\theta$}
\end{center} \end{figure}

For each cell $C$, denote the angle interval in which it exists by $\theta^{C}$. In this interval, there is a single angle at which the opening is minimum for all values of $x$ of the top edge, the angle perpendicular to the bottom edge of the cell. If we change the angle incrementally (to both directions) from that specific angle, and opening will monotonically increase. This insight allows us to limit the number of intervals in the following division: given a query measurement $d$, we divide each cell angle interval into sets of sub intervals $\theta^{C}_{1},\theta^{C}_{2},\theta^{C}_{3}$:
\[ \begin{matrix}
    \theta \in \theta^{C}_{1} \iff \forall x \in [x^{C}_{tl}(\theta),x^{C}_{tr}(\theta)], O^{C}(\theta,x) \geq d \\
    \theta \in \theta^{C}_{2} \iff \exists x_{\theta^{C}_{2}}: \forall x \in [x^{C}_{tl}(\theta),x_{\theta^{C}_{2}}], O^{C}(\theta,x) \geq d, \forall x \in (x_{\theta^{C}_{2}},x^{C}_{tr}(\theta)], O^{C}(\theta,x) < d \\
    \theta \in \theta^{C}_{3} \iff \exists x_{\theta^{C}_{3}}: \forall x \in [x^{C}_{tl}(\theta),x_{\theta^{C}_{3}}), O^{C}(\theta,x) < d, \forall x \in [x_{\theta^{C}_{3}},x^{C}_{tr}(\theta)], O^{C}(\theta,x) \geq d \\
\end{matrix} \]
Note that $\theta^{C}_{1}$ may contains up to two continues intervals of angles, and $\theta^{C}_{2},\theta^{C}_{3}$ may contain only one, due to the explanation of the minimum opening above. If there is some angle $\theta \in \theta^{C}$ such that $\theta \notin \theta^{C}_{1} \cup \theta^{C}_{2} \cup \theta^{C}_{3}$, we ignore it, it doesn't contain any points for the query result. Note that we can calculate $x_{\theta^{C}_{2}},x_{\theta^{C}_{3}}$ as a function of $\theta$. Denote $X^{C}_{l}(\theta)$, $X^{C}_{r}(\theta)$:
\[ X^{C}_{l}(\theta)=\begin{cases}
    x_{v^{C}_{l}} & \theta \in \theta^{C}_{1} \cup \theta^{C}_{2} \\
    x_{\theta^{C}_{3}}, & \theta \in \theta^{C}_{3}
\end{cases} \quad
X^{C}_{r}(\theta)=\begin{cases}
    x_{v^{C}_{r}} & \theta \in \theta^{C}_{1} \cup \theta^{C}_{3} \\
    x_{\theta^{C}_{2}}, & \theta \in \theta^{C}_{2}
\end{cases} \]

For a fixed cell $C$, fixed angle $\theta$, given a query measurement $d$, denote by $(x',y')$ a point that lays on the line $((X^{C}_{l}(\theta),e^{C}_{t}(X^{C}_{l}(\theta))), (X^{C}_{r}(\theta),e^{C}_{t}(,X^{C}_{r}(\theta))))$. The point $(x,y)=(x'-d\cos \theta,y'-d\sin \theta)$ is a possible position for the robot to measure $d$ at angle $\theta$ the edge $e^{C}_{t}$.

(TODO, not sure how to do this section) To calculate these points for all angles, we can calculate the functions that represent the endpoints of the above mentioned result line, as a function of $\theta$. still need to subtract $(d\cos \theta,d\sin \theta)$. We will have two functions $f_{1},f_{2}:\theta \rightarrow x\times y$, and all points on the straight line between them should be included in the output, not sure how to convert that to regular representation of points in 2D.

For each cell we calculated the area in 2D that represent the points that the robot might be in, and we overlay all of these areas to form the final output. $O(n^{2} \log n)$ ?.

\subsubsection*{Output Sensitive Optimizations}

Denote $O^{C}_{max}$ the maximum opening of a cell $C$, that is $\max_{\theta \in \theta ^{C}} \max_{x \in [X^{C}_{l}(\theta),X^{C}_{r}(\theta)]} O^{C}(\theta,x)$. During preprocessing, we can calculate $O^{C}_{max}$ for all cells, and sort them by it. Given a query measurement $d$, we can check only the cells that will have non empty output for the query, these are the cells that hold $d \leq O^{C}_{max}$. We can find these cells using a binary search on the sorted cells, resulting in $O(k\log k + \log n)$ query time (TODO assuming overlaying in $O(k\log k)$), where $k$ is the number of cells that contains points of output (TODO is $k$ optimal?). This method requires preprocessing of $O(n^{2}\log n)$ time and $O(n^{2})$ space.

\subsubsection*{Second Measurement Variant}

After the robot performed a single measurement $d_{1}$, we can guarantee a second measurement to a different edge by turning $180$\textdegree, denoted $d_{2}$. Using the same cells and definitions as above, for a fixed angle $\theta$, each cell $C$ contains one point the robot might be in (assuming the top and bottom edges of the cell are not parallel), and we can calculate it by solving $O^{C}(\theta,x)=d_{1}+d_{2}$. For all angles, we can calculate a function that will define these points as a function of $\theta$ (see calculation section). To apply similar output sensitive optimizations, we calculate for each cell $O^{C}_{min}$ in addition to $O^{C}_{max}$, and build an interval tree, where each entry is a cell with interval key $[O^{C}_{min},O^{C}_{max}]$. Given a measurements query $d_{1},d_{2}$, we find all the cells that satisfy $O^{C}_{min} \leq d_{1}+d_{2} \leq O^{C}_{max}$, which can be done in $O(\log n)$ using the interval tree. After finding all the relevant cells, we calculate for each one the function that define the points that robot might be in the cell, and union the results (TODO union lines? maybe just return as is). Prepossessing requires $O(n^{2} \log n)$ time and $O(n^{2})$ space, query requires $O(k + \log n)$.

\newpage{}
\subsubsection*{Calculation details}

\begin{itemize}
  \item Calculating $O^{C}(\theta, x)$:
        \begin{center} \includegraphics[scale=0.3]{img/opening_calc.png} \end{center}
        The top angle is $90-\theta$, the angle opposite of $z$ is $90 + \theta -\alpha$. Using the law of sines:
        \[ O=\frac{z\sin \alpha}{\sin (90 + \theta -\alpha)}=\frac{(e^{C}_{t}(x)-e^{C}_{b}(x))\sin \alpha}{\cos (\alpha - \theta)} \]

    \[ \text{TODO simplifications: } \alpha = \arctan (\frac{1}{m_{e_{b}}}) ,\quad \sin \arctan (\frac{1}{m_{e_{b}}}) = \frac{1}{\sqrt{m_{e_{b}}^2 + 1}} \]

  \item Calculating $x^{C}_{tl}(\theta),x^{C}_{tr}(\theta)$:
        \[ e^{C}_{t}(x)=m_{e_{t}}x+b_{e_{t}} ,\quad y=\tan \theta (x - x_{v^{C}_{i}}) + y_{v^{C}_{i}} \]
        \[ m_{e_{t}}x+b_{e_{t}}=\tan \theta x - \tan \theta x_{v_{i}} + y_{v_{i}} \]
        \[ x(m_{e_{t}} - \tan \theta)= y_{v_{i}} - \tan \theta x_{v_{i}} -b_{e_{t}} \]
        \[ x= \frac{y_{v_{i}} - \tan \theta x_{v_{i}} -b_{e_{t}}}{(m_{e_{t}} - \tan \theta)}\]
        \[ x^{C}_{tl}(\theta)=\frac{y_{v_{l}} - x_{v_{l}} \tan \theta -b_{e_{t}}}{m_{e_{t}}- \tan \theta} \quad x^{C}_{tr}(\theta)=\frac{y_{v_{r}} - x_{v_{r}} \tan \theta -b_{e_{t}}}{m_{e_{t}}- \tan \theta} \]

    \item Solving $O^{C}(\theta,x)=z$ ($z=d$ in $x_{\theta^{C}_{2}},x_{\theta^{C}_{3}}$ calculation, $z=d_{1}+d_{2}$ in "Second Measurement Variant"):
    \[ O^{C}(\theta,x)=z \]
    \[ \frac{(e^{C}_{t}(x)-e^{C}_{b}(x))\sin \alpha}{\cos (\alpha - \theta)}=z \]
    \[ e^{C}_{t}(x)-e^{C}_{b}(x))=\frac{z \cos (\alpha - \theta)}{\sin \alpha} \]
    \[ m_{e_{t}}x +b_{e_{t}}-m_{e_{b}}x -b_{e_{b}}=\frac{z \cos (\alpha - \theta)}{\sin \alpha} \]
    \[ x=\frac{z \cos (\alpha - \theta)}{(m_{e_{t}} -m_{e_{b}})\sin \alpha} + \frac{b_{e_{b}} - b_{e_{t}}}{m_{e_{t}} -m_{e_{b}}}\]
    This result is the $x$ value of the possible measurement on the top edge, to get the possible robot location one should subtract $(d_{1}\cos \theta,d_{1}\sin \theta)$.

\end{itemize}

\end{document}
